\documentclass{article}
\newcommand\tab[1][1cm]{\hspace*{#1}}
\usepackage{amsfonts}
\usepackage[margin=1in]{geometry}
\usepackage[mathscr]{euscript}
\usepackage{enumitem}
\usepackage{amsmath}
\begin{document}
\title{MA 35300: HW 9}
\author{Jason Shipp}
\maketitle
\section*{Problem 1}
\subsection*{Problem}
Let the rows of \(A \in M_{n x n}(F)\) be \(a_{1},a_{2},...,a_{n}\), and let \(B\) be the matrix in which the rows 
\\ are \(a_{n},a_{n-1},...,a_{1}\). Calculate det\((B)\) in terms of det\((A)\).
\subsection*{Answer}
The matrix B is formed by independent row changes where \(a_{n}\) swaps with \(a_{1}\), \(a_{n-1}\) swaps with \(a_{2}\), etc...
\\ These swaps occur \(\frac{n}{2}\) Times, and by Theorem 4.5 introduce a multiplication of -1 each time
\\ Therefore \(det(B) = (-1)^{\frac{n}{2}} det(A)\)
\section*{Problem 2}
\subsection*{Problem}
Use Cramer's Rule to solve the given system of linear equations:
\\ \tab[2cm] \(a_{11}x_{1} + a_{12}x_{2} = b_{1}\)
\\ \tab[2cm] \tab[2cm] \tab[2cm] where \(a_{11}a_{22} - a_{12}a_{21} \ne 0\)
\\ \tab[2cm] \(a_{21}x_{1} + a_{22}x_{2} = b_{2}\)
\subsection*{Answer}
Let \(\quad A = \begin{pmatrix} a_{11} & a_{12} \\ a_{21} & a_{22} \end{pmatrix}\)
\\ Using Cramer's Rule, namely \(x_{k} = \frac{det(M_{k})}{det(A)}\) we can form the linear equation:
\\ \(x_{1} =\frac{\quad det\begin{pmatrix} b_{1} & a_{12} \\ b_{2} & a_{22}\end{pmatrix}}{\quad det(A)} = \frac{b_{1}a_{22} 
- a_{12}b_{2}}{a_{11}a_{22}-a_{12}a_{21}}\),
\\
\\ \(x_{2} =\frac{\quad det\begin{pmatrix} a_{11} & b_{1} \\ a_{21} & b_{2}\end{pmatrix}}{\quad det(A)} = \frac{b_{1}a_{11} 
- a_{21}b_{2}}{a_{11}a_{22}-a_{12}a_{21}}\).
\\\\\\\\\\\\
\section*{Problem 3}
\subsection*{Problem}
Use Theorem 4.8 to prove a result analogous to Theorem 4.3 (page 212), but for columns:
\\ \begin{enumerate}[label=(\alph{*})]
	\item The determinant of an \(n\) x \(n\) matrix is a linear function of each column when the remaining columns are held fixed
	\item If \(A \in M_{nxn}(F)\) has a column cosisting entirely of zeroes, then det(A) = 0.
\end{enumerate}
\subsection*{Answer}
Theorem 4.8: det(\(A^{t}\)) = det(A)
\\ Theorem 4.3: The determinant of an \(n\) x \(n\) matrix is a linear function of each row when the remaining rows are held fixed. That is, for \(1 \le r \le n\), we have 
\\ \tab[2cm] det\(\begin{pmatrix}a_{1}\\ \vdots \\a_{r-1} \\u + kv \\a_{r+1} \\ \vdots \\a_{n} \end{pmatrix}\) =
	det\(\begin{pmatrix}a_{1}\\ \vdots \\a_{r-1} \\u \\a_{r+1} \\ \vdots \\a_{n} \end{pmatrix}\) +
	\( k\) det\(\begin{pmatrix}a_{1}\\ \vdots \\a_{r-1} \\v \\a_{r+1} \\ \vdots \\a_{n} \end{pmatrix}\)
\\ \begin{enumerate}[label=(\alph{*})]
\item The detirminant of an \(n\) x \(n\) matrix is a linear function of each column when the remaning columns are held fixed.
\\	When you take \(A^{t}\) the columns of A become the rows of of \(A^{t}\).
\\	Because of Theorem 4.8, Theorem 4.3 can be applied again towards \(A^{t}\) which makes the above statement true.
\item If \(A \in M_{nxn}(F)\) has a column consisting entirely of zeroes, then det(A) = 0
	\\A column of zeroes in A means a row of zeroes in \(A^{t}\)
	\\Theorem 4.4 states if a matrix, \(A^{t}\), has a row of zeroes det(A) = 0
	\\By Theorem 4.8: det(\(A^{t}\)) = det(A) = 0
\end{enumerate}
\section*{Problem 4}
\subsection*{Problem}
Prove that an upper triangular \(n\) x \(n\) matrix is invertible if and only if all of its diagonal entries are nonzero.
\subsection*{Answer}
By the proof on the final page of Monday's notes. The determinant of a matrix is equal to the product of the entries along the diagonal
\\ That is: det(A) = \(\prod\limits_{k=1}^{n} a_{kk}\) where \(a_{kk} \in A\)
\\ If one of the entries \(a_{kk}\) for \(1 \le k \le n\) is 0, then the det(A) = 0
\\ Therefore A is only invertible if the diagonal entries are nonzero.
\\ For the only if portion, if none of the entries along the diagonal are zero then the determinant is nonzero
\\ So A is invertible.
\section*{Problem 5}
\subsection*{Problem}
Let \(\beta = \{u_{1},u_{2},...,u_{n}\}\) be a subset of \(F^{n}\) consisting of n distinct vectors, and let \(B\) be the matrix in \(M_{nxn}(F)\) having \(u_{j}\) as colum \(j\). Prove that \(\beta\) is a basis for \(F^{n}\) if and only if det(B) \(\ne\) 0.
\subsection*{Answer}
If the set \(\beta\) is a basis then \(\beta\) is an indenpendent set of size n.
\\This is the same as the set of columns of \(B\) being independent
\\If the columns are independent then the rank(B) = n
\\rank(B) = n if and only if det(B) \(\ne\) 0.
\\Therefore \(\beta\) is a basis for \(F^{n}\) if and only if det(B) \(\ne\) 0
\section*{Problem 6}
\subsection*{Problem}
Prove that if \(A,B \in M_{nxn}(F)\) are similar, then \(det(A) = det(B)\). 
\subsection*{Answer}
\(B = Q^{-1}AQ\)
\\\(det(B) = det(Q^{-1}AQ)=det(Q^{-1})det(A)det(Q) = det(Q)^{-1}det(A)det(Q) = det(A)\)
\end{document}