\documentclass{article}
\newcommand\tab[1][1cm]{\hspace*{#1}}
\usepackage{amsfonts}
\usepackage[margin=1in]{geometry}
\usepackage[mathscr]{euscript}
\usepackage{enumitem}
\usepackage{amsmath}
\begin{document}
\title{MA 35300: HW 9}
\author{Jason Shipp}
\maketitle
\section*{Problem 1}
\subsection*{Answer}
\begin{enumerate}[label=(\alph{*})]
  \item
    If \(A\) is similar to \(\lambda I_{n}\), then there exists \(Q_{-1} and Q\) such that\\
    \(A = Q^{-1}\lambda I_{n}Q\) \\
    \(  = \lambda Q^{-1}I_{n}Q\) \\
    \(  = \lambda Q^{-1}Q\) \\
    \(  = \lambda I_{n}\) \(\square\)
  \item
    Let \(A\) be some diagonal matrix with elements along the diagonal \(a_{1},\dots,a_{n}\) with the rest of the elemnts being zero \\
    Then the characteristic polynomial of \(A, p(\lambda) = det(A - \lambda I_{n}) = (a_{1} - \lambda)...(a_{n} - \lambda)\)\\
    Which means that for the diagonal matrix the eigen values are the values \(a_{1},\dots,a_{n}\) \\
    This means the matrix having only one eigen value is the matrix whose diagonal is \(\lambda,\dots,\lambda\) or \\
    \(\lambda * I_{n}\) \(\square\)
  \item
    \(A - \lambda I_{n}\) = \(\quad \begin{pmatrix} 1-\lambda & 1 \\ 0 & 1 - \lambda \end{pmatrix}\) \\
    \(p_{A}(\lambda) = det(A-\lambda I_{n}) = (1-y)(1-y)\) \\
    Therefore the only eigen value of A is 1 \\
    The nullity of \(A - 1 * I_{2} = \quad \begin{pmatrix} 0 & 1 \\ 0 & 0 \end{pmatrix}\) is one.\\
    Therefore we cannot find a set of 2 independent eigenvectors for a, which means A cannot be diagonizable. \(\square\)
\end{enumerate}    
\section*{Problem 2}
\subsection*{Answer}
\begin{enumerate}[label=(\alph{*})]
  \item
    If A is similar to B then there exists \(Q^{-1} and Q\) such that\\
    \(A = Q^{-1}BQ\) \\
    \(p_{A}(\lambda) = det(A - \lambda I_{n}\) \\
    \(p_{B}(\lambda) = det(B - \lambda I_{n}\) \\
    \( = det(B - \lambda I_{n})\)\(det(I_{n}) = det(B - \lambda I_{n})(QQ^{-1})\) \\
    \( = det((B - \lambda I_{n})(QQ^{-1}) = det(Q^{-1}(B-\lambda I_{n})Q)\) \\
    \( = det(Q^{-1}BQ - \lambda I_{n})\) \\
    \( = det(A - \lambda I_{n})\) \\
    \( = p_{A}(\lambda) \) \\
    Therefore similar matrices have the same characteristic polynomial. \(\square\)
  \item
    Let Q be the matrix which takes \(\beta \to \beta'\) \\
    Let B be the matrix representation of one of the matrixes in T\\
    By it's construction \(A = Q^{-1}BQ\) \\
    Therefor A is similar to B and by part a: \\
    \(p_{A}(\lambda) = p_{B}(\lambda)\) independent  of choice of vbases \(\square\)
\end{enumerate}
\section*{Problem 3}
\subsection*{Answer}
  \(p_{A}(\lambda) = det(A - \lambda I_{n})\)\\
  \( = det(A - \lambda I_{n})^{t}\)\\
  \( = det((A - \lambda I_{n})^{t})\)\\
  \( = det(A^{t} - \lambda I_{n})\) \\
  \( = p_{A^{t}}(\lambda)\)
\section*{Problem 4}
\subsection*{Answer}
Because B is invertible we can say that \(A + cB = (B^{-1}A + cI_{n})B\)\\
\(det(A + cB) = det(B^{-1}A + cI_{n})det(B)\) \\
This means the determinant is a polynomial of c which has some countable number of zeroes \\
Therefore we can always get some value \(c \in \mathbb{C}\) such that the determinant is nonzero and \(A + cB\) is not invertible\\
\end{document}
