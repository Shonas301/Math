\documentclass{article}
\usepackage{amsfonts}
\usepackage[margin=1in]{geometry}
\usepackage[mathscr]{euscript}
\usepackage{enumitem}
\usepackage{amsmath}
\begin{document}
\title{MA 35300: HW 8}
\author{Jason Shipp}
\maketitle
\section*{Problem 1}
\subsection*{Answer}
Given matrix A, the matrix B would be the matrix: \(\quad B = \begin{pmatrix}
A_{12} & A_{11} \\
A_{22} & A_{21} \end{pmatrix} \)
\\ This means \(Det(B) = A_{12}A_{21} - A_{11}A_{22} = -1(A_{11}A_{22} - A_{12}A_{21}) = -1Det(A)\)
\section*{Problem 2}
\subsection*{Answer}
If the two columns of A are equal, that means that \(A_{11} = A_{12}, A_{21} = A_{22}\)
\\ This means their products \(A_{11}A_{22} = A_{12}A_{21} \rightarrow A_{11}A_{22} - A{12}A{21} = Det(A) = 0\)
\section*{Problem 3}
\subsection*{Answer}
Given A above, \(\quad A^{t} = \begin{pmatrix}
A_{11} & A_{21} \\
A_{12} & A_{22} \end{pmatrix} \)
\\ \(Det(A^{t}) = A_{11}A_{22} - A_{21}A_{12} = A_{11}A_{22} - A_{12}A_{21} = Det(A)\) by communativity of real numbers
\section*{Problem 4}
\subsection*{Answer}
If A is an upper triangular matrix it takes the form, \(A = \begin{pmatrix} A_{11} & A_{12} \\ 0 & A_{22}\end{pmatrix}\)
\\ The determinant of a matrix in such a form would be \(Det(A) = A_{11}A_{22} - A_{12}*0 = A_{11}A_{22}\) 
\\ which is the product of the diagonal
\section*{Problem 5}
\subsection*{Answer}
For this section let A be defined above and \(B = \begin{pmatrix} B_{11} & B_{12} \\ B_{21} & B{22} \end{pmatrix} \)
\\ The product \(AB = \begin{pmatrix}
A_{11}B_{11} + A_{12}B_{21} & A_{11}B_{12} + A_{12}B_{22} \\
A_{21}B_{11} + A_{22}B_{21} & A_{22}B_{22} + A_{21}B_{12} \end{pmatrix}\)
\\ \(Det(AB) = (A_{11}B_{11} + A_{12}B_{21})*(A_{22}B_{22} + A_{21}B_{12}) - (A_{11}B_{12} + A_{12}B_{22})*(A_{21}B_{11} + A_{22}B_{21})\)
\\ \( = (A_{11}A_{22} - A_{12}A_{21})(B_{11}B_{22} - B_{12}B_{21}) = Det(A) * Det(B)\)
\section*{Problem 6}
\subsection*{Answer}
a. \(CA = \begin{pmatrix} A_{11}A_{22} - A_{12}A_{21} & 0 \\ 0 & A_{11}A_{22}-A_{12}A_{21} \end{pmatrix}\)
\\  \(AC = \begin{pmatrix} A_{11}A_{22} - A_{12}A_{21} & 0 \\ 0 & A_{11}A_{22}-A_{12}A_{21} \end{pmatrix}\)
\\  Therefore CA = AC
\\  \(Det(A) = A_{11}A_{22} - A_{12}A_{21}\), \(Det(A)*I = \begin{pmatrix} A_{11}A_{22} - A_{12}A_{21} & 0 \\ 0 * A_{11}A_{22}-A_{12}A_{21} \end{pmatrix}\)
\\  Therefore \(CA = AC = Det(A)*I  \)
\\ \\ b.\(det(C) = A_{22}A_{11} - -A_{12}-A_{21} = A_{22}A_{11} - A_{12}A_{21} = Det(A) \)
\\ \\ c. \(A^{t} = \begin{pmatrix} A_{11} & A_{21} \\ A_{12} & A_{22} \end{pmatrix}\)
\\ classical adjoint of \(A^{t} = \begin{pmatrix} A_{22} & - A_{21} \\ -A_{12} & A_{11} \end{pmatrix} \)
\\ \(C^{t} = \begin{pmatrix} A_{22} & -A_{21} \\ -A_{12} & A_{11} \end{pmatrix}\)
therefore the classical adjoint of \(A^{t}\) is \(C^{t}\)
\\ d. Because A is invertible we can write \([det(A)]^{-1}CA = A[det(A)]^{-1}C = I \rightarrow A^{-1} = [det(A)]^{-1} C\)
\section*{Problem 7}
\subsection*{Answer}
\(det(A) = \sum\limits_{i=1}^{n+1}(-1)^{1+j}A_{1j}det([A]_{1j})\)
\\where \([A]_{1j}\) is the matrix with \(j \ne 1\) and rank less than n + 1, which has det(0)
\\\([A]_{11}\) however with \(det([A]_{11}) = \prod\limits_{i=2}^{n+1}A_{ii}\)
\\So the statement above can be rewritte \(det(A) = A_{11}det([A]_{11}) = \prod\limits_{i=1}^{n+1}A_{ii}\)
\\Which is the product along the diagnol
\section*{Problem 8}
\subsection*{Answer}
Use theorem 4.3 and apply it to each row such that \\
\(det(ka_{1},ka_{2},...,ka_{n}^{t} = kdet(a1, ka_{2},...,ka_{n})^{t} = .. = k^{n} (a_{1},a_{2},...,a_{n})\)
\section*{Problem 9}
\subsection*{Answer}
If A has two identical columns one of the columns can be set to 0, and therefore the rank of the matrix is one less than it should be.
\\If rank is not max for the matrix then det(A) = 0
\end{document}