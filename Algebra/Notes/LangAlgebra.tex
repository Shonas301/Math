\documentclass{article}
\newcommand\tab[1][1cm]{\hspace*{#1}}
\newcommand*\Heq{\ensuremath{\overset{\kern2pt H}{=}}}
\usepackage{amsfonts}
\usepackage[margin=1in]{geometry}
\usepackage[mathscr]{euscript}
\usepackage{enumitem}
\usepackage{amsmath}
\begin{document}
\title{Algebra Notes}
\author{Jason Shipp}
\maketitle

\section*{Chapter 1: Groups}

\subsection*{MONOIDS}
    Let S be a set, with mapping \(S x S \to S\)
    an element of \(e \in S\) such that \(ex = x = xe\) for all
    \(x \in S\) is called a unit element. When it is defined
    for Addition e = 0, the zero element
    A unit element is unique if for e` is a another unit
    element, we have \(e = ee` = e`\) by assumption
    in most cases e is just written as 1, but e is used
    for more basic properties
    \\
    A monoid is a set G, with the law of composition
    which is associative, and having a unit element
    (therefore \(card(G) \ne 0\)).
    \\  
    Let G be a monoid, and x1, ..., xn elements of G
    (where \(n \in Z+\backslash{1})\) \\
    We define their product inductively: \\
        \(\prod\limits_{v=1}^{n} x_{v}\)\\
    We then have the following rule: \\
        \(\prod\limits_{u=1}^{m}x_{u} * \prod^{n}x_{m+v} =
        \prod\limits_{v=1}^{m+n}x_{v}\)\\
    Which asserts that we can insert parentheses in any
    manner in our product without changing the value
    As a matter of convention when n = 0, the product
    equals e.
    \\
    Under more general maps like \(S1 x S2 \to S3\) we need
    more general ways to express associativity and 
    commutativity for any setting in which they make 
    sense. 
    \\
    Commutativity means \(f(x,y) = f(y,x)\), or \(xy = yx\)
    If the composition for G is commutative we say that
    G is abeilian.

    Let G be a commutative monoid, and x1,...,xn elements
    of G. let \(\psi\) be a bijection of the set of integers
    (1,...,n) onto itself. Then:\\
        \(\prod x_{\psi(v)} = \prod x_{v}\)\\
    This essentially just means we can use the operation
    on the elements of G in any order.

    Let G be a commutative monoid, let I be a set, and
    let \(f: I \to G\) be a mapping such that \(f(i) = e\) for
    all but a finite number of \(i \in I\).
    Let I0 be the subset of I consisting of those I
    such that \(f(i) \ne e\). By\\
        \(\prod f(i)\) we shall mean the product \\
        \(\prod f(i)\) for \(i \in I_{0}\)\\
    If G is written additively use \(\sum\) instead\\
\end{document}
