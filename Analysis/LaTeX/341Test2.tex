------ Continuity ------

Definition 3.1 (Continous Function):
    Given topological spaces (X, Tx), (Y, Ty), we say that a function
    f: X -> Y is continuous at x \in X if for each open neighborhood
    V \subset Y of y = f(x), there exists an open neighborhood
    U \subset X of x such that f(U) \subset V.

If you view the points of V has the points in Y that are close to y,
then continuity says that you can find some open set U \subset X
that contains x that if x` \in U then f(x`) = y` \in V

If X,Y are metric spaces with dx, dy, we can state the continuity of
f at x follows:
    Given an open ball B(y, ry) there exists an open ball B(x, rx)
    such that if x` \in B(x,rx) then f(x`) \in B(y,ry)

Remark 12:
    For continuity at x with f(x) = y, one only needs to check V 
    that are in a neighborhood sub-base of the set of open
    neighborhoods of y.
    Given an open subset U \subset X and a function f: X -> Y, we 
    say that f is continuous on U if f is continous at each x \in U.
    When U = X, we simply say that f is continuous.

Lemma 3.2:
    If (X, Tx), (Y, Ty) are topological spaces. f: X -> Y is 
    continous at x \in X and {x_n} is a sequence in X that converges
    to x, then {f(x_n)} is a convergent sequence and {f(x_n)} 
    converges to f(x)

Proposition 3.3:
    Let (X, Tx), (Y, Ty) be topological spaces and f: X -> Y a 
    function. Then the following are equivalent:

        (a) f is continuous
        (b) For each open subset V \subset Y, the subset f^-1(V) is
            open in X.
        (c) For each closed subset C \subset Y, the subset f^-1(C)
            is closed in X.
Lemma 3.4:
    Let (X, Tx), (Y, Ty), and (Z, Tz) be topological spaces and 
    f: X -> Y, g: Y -> Z be functions. If f is continuous at x and g
    is continuous at f(x), then g of f is continuous at x. In 
    particular, if f,g are continuous, then g of f is continous.
