\documentclass{article}
\newcommand\tab[1][1cm]{\hspace*{#1}}
\newcommand*\Heq{\ensuremath{\overset{\kern2pt H}{=}}}
\usepackage{amsfonts}
\usepackage[margin=1in]{geometry}
\usepackage[mathscr]{euscript}
\usepackage{enumitem}
\usepackage{amsmath}
\begin{document}
\title{MA 34100: Midterm 2}
\author{Jason Shipp}
\maketitle

\section{}

\subsection*{Problem}
Let \(f:[a,b] \to \mathbb{R}\) be a constant function \(f(x) = x_{x}\) Using the definition of Reimann integrable to prove that f is integrable on 
\([a,b]\) \(\int^{b}_{a}f = x_{0}(b-a)\).

\subsection*{Answer}
Because \(f\) is constant to some value \(\alpha \in \mathbb{R}\) we know
that \(U(f, [a,b],\mathcal{P}) \leq M(f)(b-a)\) where \(M(f) = \alpha\)
and \(L(f, [a,b], \mathcal{P}) \geq m(f)(b-a)\) where \(m(f) = \alpha\)
therefore \(U(f, [a,b],\mathcal{P}) = L(f, [a,b],\mathcal{P})\) by lemma 3.5
which means \(\int^{-}_{[a,b]} = \int^{+}_{[a,b]}\) and f is integrable. \\
Furthermore because of the inequality above we know that \(M(f) = \alpha = 
supL(f, [a,b], \mathcal{P}\) and by the definition of Riemann integrable
\(\int_{a}^{b} f = \alpha(b-a)\) \tab \(\square\)


\section{}

\subsection*{Problem}
Let \(f:[a,b] \to \mathbb{R}\) be Riemann integrable. for \(\alpha \in 
\mathbb{R}\), define \(g(x) = f(x-\alpha)\). Prove that \(g\) is Riemman
intergrable on \(a + \alpha, b + \alpha]\) and \(\int^{b+\alpha}_{a+alpha}g = \int^{b}_{a}f.\)

\subsection*{Answer}
\(\int^{b+\alpha}_{a + \alpha}g(x) = \int^{b+\alpha}_{a + \alpha}f(x-\alpha)
= F(b + \alpha - \alpha) - F(a + \alpha - \alpha) = F(b) - F(a) = 
\int^{b}_{a}f\) \tab \(\square\)


\section{}

\subsection*{Problem}
Let \(f: [-a,a] \to \mathbb{R}\) be an integrable funtion. Prove that if 
\(f\) is an even function (i.e \(f(-x) = f(x)\) for all \(x\)) then 
\(\int^{a}_{-a} = 2\int^{a}_{0}f.\)

\subsection*{Answer}
\(\int^{a}_{-a}f = \int^{0}_{-a}f + \int^{a}_{0}f\) \\
\(\int^{0}_{-a}f = F(0) - F(-a) = -(F(-a) - F(0)) = -\int^{-a}_{0}f\)\\
\(\int^{0}_{-a}f = \int^{a}_{0}f(-x)\), by substitution \\
\(\int^{0}_{-a}f = \int^{a}_{0}f\) therefor \\
\(\int^{a}_{-a}f = 2\int^{a}_{0}f\) \tab \(\square\)


\section{}

\subsection*{Problem}
Let \(f:[-a,a] \to \mathbb{R}\) be an integrable function. Prove that if 
\(f\) is an odd function (i.e \(f(-x) = -f(x)\) for all of \(x\)) then 
\(\int^{a}_{-a}f = 0.\)

\subsection*{Answer}
\(\int^{a}_{-a}f = \int^{0}_{-a}f + \int^{a}_{0}f\) \\
\(\int^{0}_{-a}f = -\int^{-a}_{0}f\) \\
\(\int^{0}_{-a}f = \int^{a}_{0}f(-x)\) \\
\(\int^{0}_{-a}f = -\int^{a}_{0}f\) \\
\(\int^{a}_{-a}f = -\int^{a}_{0}f + \int^{a}_{0}f\) 
\(\int^{a}_{-a}f = 0\) \tab \(\square\)

\section{}

\subsection*{Problem}
Prove that if \(f:[a,b] \to \mathbb{R}\) is Riemann integrable and
\(a = x_{0} < x_{1} < x_{2} < \dots < x_{n} < x_{n+1} = b\), then \\
\tab[4cm] \(\int_{a}^{b}f = \sum\limits^{n}_{i=0}\int^{x_{i+1}}_{x_{i}}f\).

\subsection*{Answer}
\(\int_{a}^{b}f = \sum\limits^{n}_{i=0}\int^{x_{i+1}}_{x_{i}}f\) \\
\(\int_{a}^{b}f = \sum\limits^{n}_{i=0}(F(x_{i+1}) - F(x_{i}))\) \\
\(\int_{a}^{b}f = (F(x_{1}) - F(x_{0})) + (F(x_{2}) - F(x_{1})) + \dots +
(F(x_{n}) - F(x_{n-1})) + (F(x_{n+1}) - F(x_{n})) \) \\
\(\int_{a}^{b}f = F(x_{n+1}) - F(x_{0}) = F(b) - F(a) = \int_{a}^{b}f\)
\tab \(\square\)

\section{}

\subsection*{Problem}
Prove that if \(f, g: [a,b] \to \mathbb{R}\) are Reiemann integrable, then 
\(f+g\) is a Riemann ingegrable and \\
\tab[4cm] \(\int^{b}_{a}(f+g) = \int^{b}_{a}f + \int^{b}_{a}g\).

\subsection*{Answer}
\(U(f+g, [a,b], \mathcal{P}) = \sum\limits^{n}_{i=0} sup(f + g) 
\delta(I_{i})\) \\
\(U(f+g, [a,b], \mathcal{P}) \leq \sum sup(f)\delta(I_{i}) + \sum sup(g)
\delta(I_{i})\) \\
\(U(f+g, [a,b], \mathcal{P}) \leq U(f, [a,b], \mathcal{P}) + 
U(g, [a,b], \mathcal{P})\) \\
By the definition of the upper sum there exists the partion p such that \\
\(U(f, [a,b], \mathcal{P}) < \int_{[a,b]}^{+}f + \epsilon/2\) and
\(U(g, [a,b], \mathcal{P}) < \int_{[a,b]}^{+}g + \epsilon/2\) for \(\epsilon > 0\)\\
\(\int_{[a,b]}^{+}f+g \leq U(f+g, [a,b], \mathcal{P}) \leq +
\leq U(f, [a,b], \mathcal{P}) + U(g, [a,b], \mathcal{P}) \leq +
\int_{[a,b]}^{+}f + \int_{[a,b]}^{+}g + \epsilon\) \\
Therefore \(\int_{[a,b]}^{+}f+g \leq 
\int_{[a,b]}^{+}f + \int_{[a,b]}^{+}g + \epsilon\) \\
Following the same logic, which is difficult to type out on latex \\
\(\int_{[a,b]}^{-}f+g \geq \int_{[a,b]}^{-}f + \int_{[a,b]}^{-}g -
\epsilon\) \\
\(\int_{[a,b]}^{-}f+g \geq \int_{[a,b]}^{-}f + \int_{[a,b]}^{-}g\) \\
By definition of f and g as integrable we know:\\
\(\int_{[a,b]}^{-}f + \int_{[a,b]}^{-}g = \int_{[a,b]}^{+}f + 
\int_{[a,b]}^{+}g\) \\
And with our previous inequalities above we can rearrange: \\
\(\int_{[a,b]}^{+}f+g \leq \int_{[a,b]}^{+}f + \int_{[a,b]}^{+}g =
\int_{[a,b]}^{-}f + \int_{[a,b]}^{-}g \leq \int_{[a,b]}^{-}f+g\) \\
Which implies \(\int_{[a,b]}^{+}f+g \leq \int_{[a,b]}^{-}f+g\) and
\(\int_{[a,b]}^{-}f+g \leq \int_{[a,b]}^{+}f+g\) by definition so the 
original presented integral is indeed integrable and it is equal to 
\(\int_{a}^{b}f + \int_{a}^{b}g\) because the inequality above is equal throught due to abritrary choice of \(\epsilon\)



\section{}

\subsection*{Problem}
Prove that if \(f:[a,b] \to \mathbb{R}\) is Riemann integrable and 
\(\lambda \in \mathbb{R}\), then \(\lambda f\) is Riemann integrable and
\tab[4cm] \(\int_{a}^{b}\lambda f = \lambda \int_{a}^{b}f\)

\subsection*{Answer}
\(U(\lambda f, [a,b], \mathcal{P}) = \sum sup(\lambda f) \delta(I) =
\lambda \sum sup(f) \delta(I) = \lambda U(f, [a,b], \mathcal{P})\) \\
\(\int_{[a,b]}^{+}\lambda f = inf(\lambda U(f, [a,b], \mathcal{P})) =
\lambda inf(U(f, [a,b], \mathcal{P}))\)\\
Similarily for \(\int_{[a,b]}^{-}\lambda f = \lambda sup(L(f, [a,b],\mathcal{P}))\)  
Therefore \(\int_{[a,b]}^{-}\lambda f = \lambda sup(L(f, [a,b],\mathcal{P})) = \lambda inf(U(f, [a,b], \mathcal{P})) = \int_{[a,b]}^{+}\lambda f\)
\\ Therefore it is Reimann integrable and the equality holds because
\(\lambda \int_{a}^{b}f = \lambda \int_{[a,b]}^{+}f = \lambda inf(U(f)) = \lambda sup(L(f)) = \lambda \int_{[a,b]}^{-}f = \int_{a}^{b}\lambda f\)
\(\square\) 


\end{document}

