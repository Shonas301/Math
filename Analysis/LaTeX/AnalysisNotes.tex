\documentclass{article}
\newcommand\tab[1][1cm]{\hspace*{#1}}
\newcommand*\Heq{\ensuremath{\overset{\kern2pt H}{=}}}
\newcommand*\Def{\ensuremath{\overset{\kern2pt def}{=}}}
\usepackage{amsfonts}
\usepackage[margin=1in]{geometry}
\usepackage[mathscr]{euscript}
\usepackage{enumitem}
\usepackage{amsmath}
\begin{document}
\title{Notes on Dr. McReynold's Analysis}
\author{Jason Shipp}
\maketitle
\section*{Chapter 5}
\subsection*{Series}
\subsubsection*{Series: The basics}
Given a finite set \(A \subset \mathbb{R}\), we can list off the
elements of \(A = \{a_{1},...,a_{n}\}\). We are going to allow
seperate \(a_{i} = a_{j}, i \ne j\)\\
We can view this data as a function 
\(f: {1, ..., n} \to \mathbb{R}\) where \(f(i) = a_{i}\).
The sum of a is:\\
\tab[4cm]   \(S(A) \Def a_{1} + ... + a_{n} = 
            \sum\limits_{i=1}^{n}a_{i}\)\\
\end{document}
